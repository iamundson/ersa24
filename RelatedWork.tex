%related work
Research on applying generative AI to formal reasoning has gained significant attention. For instance, OpenAI researchers conducted pioneering work in 2020, leveraging large language models (LLMs) for mechanical theorem proving \cite{polu2020generative}. This resulted in the development of GPT-f, a proof assistant for Metamath, which achieved a 56\% success rate and proved 200 theorems \cite{megill2019metamath}. Other studies have explored LLMs for proof generation and repair. First et al. achieved a 50\% success rate in proof repair for Isabelle/HOL \cite{first2023baldur}, using Minerva \cite{lewkowycz2022solving}, a model based on Google’s PaLM \cite{chowdhery2022palm}. Research has also examined GPT-3.5 and GPT-4 for Coq theorem proving \cite{zhang2023getting}, primarily focusing on diagnosing failed proofs. LLMs have further been applied to discover program invariants \cite{pei2023can, wu2023lemur} and support automated reasoning, as seen in the Clover project by Stanford and VMware, which emphasizes verifiable code generation \cite{sun2024clover}.
%
In previous work \cite{CoqDog} \cite{CoqDogHCSS24} Tahat et al. developed a copilot for large-scale proof repair using a multi-shot conversational learning approach. The system achieved a 97\% success rate across 58 theorems from a repository containing 20,000 lines of Coq code from the Copland proofbase \cite{CoqDog}. Additionally, they introduced an evaluation framework to assess the convergence of dialogues toward predefined proof sets.
%
While this paper builds on \cite{CoqDog}
success, it represents a paradigm shifts to model repair in the context of model-based systems engineering (MBSE) for AADL models. %It integrates human requirements inputs and context retrieval-augmented generation methods to address and resolve counterexamples detected by the AGREE model checker. 
To the best of our knowledge, this is the first application of generative AI for counterexample explanation and repair in AGREE model checking and AADL models.
