% PROVERS
PROVERS OVERVIEW - 1 paragraph

% INSPECTA
To address these challenges, our team is developing the Industrial-Scale Proof-Engineering for Critical Trustworthy Applications (INSPECTA) framework.  INSPECTA consists of \textit{ProofOps} and \textit{BuildOps} tools and methods that integrate with current aerospace DevOps pipelines and achieve provably correct design and implementation at each level of the system hierarchy.  In order to address the key objectives of PROVERS, we pay particular attention to addressing scalability and explainability concerns with respect to the proof tools in our framework.

% Compositional Reasoning
Within the ProofOps workflow, INSPECTA uses the Assume-Guarantee Reasoning Environment (AGREE)~\cite{}, a formal compositional reasoning tool for Architecture Analysis and Design Language (AADL)~\cite{} models.
Compositional reasoning~\cite{} partitions the formal analysis of a complex system architecture into verification tasks corresponding to the architecture's decomposition.  By partitioning the verification effort into proofs about each subsystem within the architecture, the analysis will scale to handle large system designs.

Although AGREE does not suffer from some of the scalability issues inherent in other formal methods frameworks due to the compositional nature of the analysis, generated counterexamples can still be difficult to understand, especially for formal methods novices when the counterexamples contain several step, each consisting of multiple variables.  This problem is not unique to AGREE, but is common to most model checkers in use today~\cite{}.  Recently, a novel approach to producing explainable counterexamples has emerged in the form of generative AI.

In this paper, we present our current work on using generative AI to provide clear and concise explanations of counterexamples generated by AGREE.  Our initial results indicate this approach is well-suited to providing clear explanations of root cause, and even provide suggestions for addressing the contract violations.